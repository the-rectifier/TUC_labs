\documentclass[12pt, a4paper]{article}
\usepackage[margin=3cm]{geometry}
\usepackage[utf8]{inputenc}
\usepackage{color, amssymb}
\usepackage{listings, amsmath, float}
\renewcommand{\baselinestretch}{1.5}
\setlength\parindent{24pt}
\usepackage{graphicx}
\graphicspath{ {./images/} }

\title{Lab 1}
\author{Odysseas Stavrou}
\date{October 2020} 

\begin{document}
\noindent\rule{\textwidth}{1.5pt}

\begin{center}
{\bf Digital Signal Processing} \\ 
 1st Lab Exercise\\
 Odysseas Stavrou 2018030199\\
 Lab Group No: 90\\
 November 2020\\
 Technical University of Crete\\
\end{center}
\noindent\rule{\textwidth}{1.5pt}


\begin{enumerate}
    \item[1.]~
    \begin{enumerate}
        \item[A.]Convolution of 2 discrete signals with and without the conv() function.
        The 2 signals chosen were 2 discrete pulses \(x_1\) and \(x_2\):
        \[x_1 = u[n-1] - u[n-3]\]
        \[x_2 = u[n-3] - u[n-5] + u[n-7] - u[n-9]\]
        \[n \in [\,0,9.8]\,\ with\ a\ step\ of\ 0.2\]

        \begin{figure}[H]
            \centering
            \includegraphics[width=\textwidth, height=9cm]{A_1.png}
            \caption{The 2 signals \(x_1\) and \(x_2\) and their convolution, from scratch and by using MATLAB's built in function, respectively}
        \end{figure}

        \item[B.]Proof of properties of Convolution and the Fourier Transform
        \[x_1[n] \circledast x_2[n] = X_1[N] \cdot X_2[N]\]
        \begin{figure}[H]
            \centering
            \includegraphics[width=\textwidth, height=10cm]{B_1.png}
            \caption{Multiplying the FFTs of both signals vs.\ taking the FFT of their convolution from earlier}
        \end{figure}
        
    \end{enumerate}
    \item[2.] 
    Fourier Transform (on paper) the following signal and sample it using the following frequencies:
    \[x(t) = 5\cos(2\pi12t) - 2\sin(2\pi\frac{3}{4}t)\]
    \begin{enumerate}
        \item[a.] \(f_s = 48Hz\)
        \item[b.] \(f_s = 24Hz\)
        \item[c.] \(f_s = 12Hz\)
        \item[d.] \(f_s = 90Hz\)
    \end{enumerate}
    \pagebreak

    \begin{enumerate}
        \item[i.] To calculate the Nyquist frequency we need to take the largest frequency (\(f_{\max}\)) of our two signals in this case \(12Hz\).
        Nyquist's frequency is the lowest sampling frequency of a signal that can reconstruct the original. Sampling with any 
        frequency lower than this, will render the reconstruction worthless.
        \[f_{nyq} = 2 * f_{\max} = 2 * 12 Hz = 24Hz\]

        \item[ii.] Using the complex identities of the sine and cosine functions we can easily derive the Fourier Transform of the above mentioned signal:
        \[\cos(t) = \frac{e^{jt} + e^{-jt}}{2}\]
        \[\sin(t) = \frac{e^{jt} - e^{-jt}}{2j}\]
        \[x(t) = \frac{5}{2}e^{2\pi j12t} + \frac{5}{2}e^{-2\pi j12t} + je^{-2\pi j\frac{3}{4}t}
        -je^{2\pi j\frac{3}{4}t}\]

        
        \item[iii.] Transforming each term from above and using the time/frequency shift properties of the FT we end up with:
        \[F\{1\} = \delta \]
        \[X(F) = \frac{5}{2}\delta(F-12) + \frac{5}{2}\delta(F+12) + 
        j\delta(F+\frac{3}{4}) - j\delta(F-\frac{3}{4})\]

        \item[iv.] As seen below, sampling with \(f_{\max}\) does not produce enough information for us to 
        be able to reconstruct the signal, where as in sampling with twice the \(f_{\max}\) yields just bearly enough information
        about the signal. Taking samples with any frequency \(>f_{\max}\), will result in a better resolution and of course 
        more samples. In my case the last sampling frequency is \(90Hz\) which is more that enough.

        \begin{figure}[H]
            \centering
            \includegraphics[width=\textwidth, height=10cm]{2.png}
            \caption{Sampling of \(x(t)\) using different frequencies}
        \end{figure}
    \end{enumerate}
    \pagebreak
    \item[3.]~
    \begin{enumerate}
        \item[A.] Capture 128 samples of the following signal and display the signal into the frequency spectrum
        \[x(t) = 10\cos(2\pi20t) - 4\sin(2\pi40t)\] with a frequency such that, the aliasing effect is not visible.

        Using: 
        \[f_{sampl} \geqslant f_{nyq} = 2 * f_{\max} \Rightarrow f_{sampl} \geqslant 80 Hz\]
        Chosen:
        \[f_{sampl} = 500 Hz\]

        We can observe peaks in the frequency spectrum at the positive and negative frequencies of the two signals 
        used to create \(x(t)\).\\
        Peaks at: 
        \begin{enumerate}
            \item[i.] \(-f_1 = -20Hz\), \(f_1 = 20Hz\)
            \item[ii.] \(-f_2 = -40Hz\), \(f_2 = 40Hz\)
        \end{enumerate}
        \begin{figure}[H]
            \centering
            \includegraphics[width=\textwidth, height=10cm]{3_A.png}
            \caption{Sampling of \(x(t)\) and its frequency spectrum}
        \end{figure}
        \item[B.] Define the following signal:
        \[x(t) = \sin(2\pi f_0t + \phi)\]
        Sampling the above signal with a SF (\(f_{s}\)) of 8KHz and by using this property:
        \[x[n] = x_a(nT_s) = x_a(n\frac{1}{f_s})\]
        the resulting discrete signal is:
        \[x[n] = \sin(2\pi \frac{f_0}{f_{s}} n + \phi)\]
        \item[i.] Plotting for all 4 (low) different sinusodial frequencies we get the following spectrum for each:   
        \begin{figure}[H]
            \centering
            \includegraphics[width=\textwidth, height=10cm]{3_B_1.png}
            \caption{Frequency Spectrum of \(x(t)\)}
        \end{figure}
    \end{enumerate}
    We can observe that, because the SF, \(f_s\), is so high in respect to \(f_0\) the peaks are shown at every \(f_0\)
    point on the Frequency Spectrum which that is expected, since that is the Frequency of our sinusodial wave.
    \pagebreak
    \begin{enumerate}
        \item[ii.]Plotting for all 4 (high) different sinusodial frequencies we get the following spectrum for each:   
        \begin{figure}[H]
            \centering
            \includegraphics[width=\textwidth, height=10cm]{3_B_2.png}
            \caption{Frequency Spectrum of \(x(t)\)}
        \end{figure}
    \end{enumerate}
    However in this case the peaks exist at points \(f_0 - f_s\). This is happening because the SF \(f_s\) is bellow the Nyquist
    limit. This is called the aliasing effect, because the sinusodial wave, clearly, has a way larger Frequency than before but, despite 
    that, the signal that ``seems'' to be sampled is a signal with a Frequency of \(f_0 - f_s\) and not \(f_0\).
    Changing the starting phase \(\phi \) will have no effect what so ever, because \(\phi \) is just a starting offset, and does not mess with 
    the Frequency.
\end{enumerate}
\end{document}